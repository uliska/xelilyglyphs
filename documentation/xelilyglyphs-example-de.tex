%%%%%%%%%%%%%%%%%%%%%%%%%%%%%%%%%%%%%%%%%%%%%%%%%%%%%%%%%%%%%%%%%%%%%%%%%%
%                                                                        %
%      This file is part of the 'xelilyglyphs' LaTeX package.            %
%                                ==========                              %
%                                                                        %
%              https://github.com/openlilylib/xelilyglyphs               %
%                                                                        %
%  Copyright 2012-2013, 2019 Urs Liska and others, ul@openlilylib.org    %
%                                                                        %
%  'xelilyglyphs' is free software: you can redistribute it and/or modify%
%  it under the terms of the LaTeX Project Public License, either        %
%  version 1.3 of this license or (at your option) any later version.    %
%  You may find the latest version of this license at                    %
%               http://www.latex-project.org/lppl.txt                    %
%  more information on                                                   %
%               http://latex-project.org/lppl/                           %
%  and version 1.3 or later is part of all distributions of LaTeX        %
%  version 2005/12/01 or later.                                          %
%                                                                        %
%  This work has the LPPL maintenance status 'maintained'.               %
%  The Current Maintainer of this work is Urs Liska (see above).         %
%                                                                        %
%  This work consists of the files listed in the file 'manifest.txt'     %
%  which can be found in the 'license' directory.                        %
%                                                                        %
%  This program is distributed in the hope that it will be useful,       %
%  but WITHOUT ANY WARRANTY; without even the implied warranty of        %
%  MERCHANTABILITY or FITNESS FOR A PARTICULAR PURPOSE.                  %
%                                                                        %
%%%%%%%%%%%%%%%%%%%%%%%%%%%%%%%%%%%%%%%%%%%%%%%%%%%%%%%%%%%%%%%%%%%%%%%%%%

\documentclass[oneside,11pt]{article}
\usepackage{xelilyglyphsStyle}
\usepackage{xelilyglyphsManualFonts}

\usepackage{polyglossia}
\setmainlanguage{german}

\pagestyle{empty}


% make the glyphs lighter
\lilyOpticalSize{26}

\begin{document}
\begin{center}
{ \Huge \xelilyglyphs }

\bigskip
{ \Large Urs Liska }

\emph{September 2013}

\end{center}

\bigskip

Sie schreiben Texte über Musik, vielleicht als Musikwissenschaftler, Lehrer oder Komponist?
Sie bereiten solche Dokumente zum Druck vor und haben schon immer die Möglichkeit vermisst, Dinge wie das Folgende zu setzen?

\begin{quote}
„In T.\,24 gilt das \decrescHairpin{} von der 2.\,\halfNote{} bis zum \lilyDynamics{sf} auf dem 11.\,\semiquaverDown[raise=-.5].“
\end{quote}

Das neue Paket \xelilyglyphs{}%
\footnote{\url{http://www.openlilylib.org/xelilyglyphs} -- \href{mailto:info@openlilylib.org}{info@openlilylib.org}}
erlaubt es auf komfortable Weise, Notationselemente aus LilyPond%
\footnote{\url{http://www.lilypond.org}}
in Textdokumenten zu verwenden.
Vorzeichen wie \flat{} oder \sharp, aber auch Zeichen wie \hspace{.5ex}\fermata{} und Taktangaben wie \lilyTimeCHalf{} oder \lilyTimeSignature{5\,+\,7}{8} sind unmittelbar zugänglich.
Aber Sie können auch beliebige Notationskonstrukte wie \lilyFancyExample{} dieses sinnfreie Beispiel in den Fließtext einfügen oder gar eingescannte Bilddateien als „Zeichen“ verfügbar machen.
Dieses Paket kann Ihre typographischen Optionen für das Schreiben oder Setzen von kritischen Berichten, analytischen Texten oder Unterrichtsmaterial erheblich erweitern.

Ein Aspekt, der \xelilyglyphs{} von allen mir bekannten Lösungen unterscheidet, ist, dass es nicht auf einen Satz vorgefertigter Symbole beschränkt ist, sondern \emph{jegliche} Notation setzen kann, die mit LilyPond realisierbar ist.

\footnotesize
Die andere Besonderheit ist die \halfNoteRest{} automatische Größenanpassung an die umgebende Schriftgröße,
\large was es leicht macht, die \halfNoteRest{} Zeichen im Fließtext zu verwenden.
\normalsize
Sie passen sich grundsätzlich von selbst an, können aber auch \clefF[scale=.4] manuell skaliert werden, einzeln oder pauschal \clefG[scale=1.3,raise=-2.9].

\medskip
Zu schön um wahr zu sein?
Zugegeben, die Sache hat einen Haken: \xelilyglyphs{} ist ein \LaTeX{}%
\footnote{\url{http://www.latex-project.org}}-Paket und erfordert daher möglicherweise ein Umdenken.
Sollten die obigen Beispiele aber Ihr Interesse geweckt haben oder Ihren professionellen Bedürfnissen entgegenkommen, dann machen Sie doch einfach einmal einen Versuch -- es ist ohnehin alles Freie Software.
(Vielleicht ist auch mein Aufsatz über textbasiertes Arbeiten eine hilfreiche Lektüre%
\footnote{\url{http://lilypondblog.org/2013/07/plain-text-files-in-music/}}).

\end{document}
